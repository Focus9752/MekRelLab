\documentclass[hidelinks]{article}
\usepackage[utf8]{inputenc}
\usepackage{float}
\usepackage[danish]{babel}
\usepackage{graphicx}
\usepackage{siunitx}
\usepackage[margin=1.25in]{geometry}
\usepackage{hyperref}
\usepackage{url}
\usepackage{csquotes}

\usepackage[
style=alphabetic,
backend=biber,
sorting=nyt
]{biblatex}

\addbibresource{references.bib}

\setlength\parindent{0pt}

\title{Labuge 9/10 - Mål bilers fart}
\author{Marcus Koschmieder Krogsgaard}
\date{15. november 2023}

\begin{document}

\maketitle

\section{Formål}
Formålet med øvelsen var at bestemme middelfarten af biler på en bestemt strækning.
\section{Fremgangsmåde}
Bilernes fart blev målt i samarbejde med Christian Ringsing, Mie Gravgaard Lassen og Veronika Mantzius Postgaard.\\
\\En person placeres mellem to tydelige markeringer på vejen (ved dette forsøg blev der benyttet to parkeringsbåse, se fig. \ref{fig:pplads}). Afstanden mellem markeringerne opmåles. Vha. et Python-script\cite{timescript} gemmes tidspunkterne hvor en given bils forhjul passerer hver markering. Trækkes den første tid fra den anden fås tiden bilen er om at krydse strækningen mellem de to markeringer, hvilket gør det muligt at finde bilens gennemsnitsfart over strækningen med følgende formel:

\[v = \frac{d}{t}\]

hvor $d$ er afstanden mellem markeringerne og $t$ er tiden bilen var om at krydse strækningen.

\begin{figure}[H]
    \centering
    \includegraphics[width = 0.5\textwidth]{img/pbåse.jpg}
    \caption{}
    \label{fig:pplads}
\end{figure}


\section{Referencer}
\printbibliography[
heading=none
]

\end{document}
